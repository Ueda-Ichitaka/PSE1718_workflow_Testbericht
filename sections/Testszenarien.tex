\chapter{Testszenarien}

    \section{Einleitung}
    In diesem Kapitel wird Bezug auf die Testszenarien genommen, welche im Pflichtenheft definiert wurden. Hierbei werden die einzelnen Szenarien erneut aufgelistet und erläutert.

            \section{Normaler Anwendungsverlauf}
        \begin{itemize}
            \item Nutzer loggt sich ein. [TF100]:
            \vspace{1mm}\\Eine Benutzerauthentifikation wurde implementiert und getestet.
            \item Nutzer erstellt / bearbeitet Workflows. [TF30], TF10], [TF20]:
            \vspace{1mm}\\Die Erstellung und das Bearbeiten von Workflows wurde implementiert und getestet.
            \item Nutzer führt Workflow aus. [TF80]:
            \vspace{1mm}\\Die Ausführung von Worklows auf WPS Servern, welche die entsprechenden WPS Prozesse anbieten, wurde implementiert und getestet. 
            \item Workflow wird im Hintergrund (Serverseitig) ausgeführt. [TF80]:
            \vspace{1mm}\\Die Ausführung von Workflows läuft im Hintergrund (Serverseitig). Hierbei schickt unser Server die einzelnen Tasks an einen WPS Server und empfängt nach der Ausführung das Ergebnis.
            \item Nutzer meldet sich ab. [TF100]:
            \vspace{1mm}\\Eine Benutzerauthentifikation wurde implementiert und getestet.
            \item Nutzer meldet sich wieder an um Status des Workflows abzufragen, bzw. Ergebnisse zu betrachten. [TF100], [TF90]:
            \vspace{1mm}\\Der Status der Workflows können durch drücken des \glqq WORKFLOWS\grqq{} Buttons in der Titelleiste angezeigt werden. 
        \end{itemize}
        \section{Anwendungsverlauf mit Verbindungsabbruch}
        \begin{itemize}
            \item Nutzer loggt sich ein. [TF100]:
            \vspace{1mm}\\Eine Benutzerauthentifikation wurde implementiert und getestet.
            \item Nutzer erstellt / bearbeitet Workflows. [TF30], TF10], [TF20]:
            \vspace{1mm}\\Die Erstellung und das Bearbeiten von Workflows wurde implementiert und getestet.
            \item Die Verbindung zum Nutzer bricht ab.
            \item Unvollständig bearbeiteter Workflow wird gespeichert. [TF70]:
            \vspace{1mm}\\Das System speichert konstant den Workflow auf dem Server.
            \item Nutzer meldet sich wieder an, und wird gefragt ob er weiterarbeiten möchte. [TF70]:
            \vspace{1mm}\vspace{1mm}\\Es wurde implementiert und getestet, dass Nutzer sich wieder anmelden und anschließend auf \glqq WORKFLOWS\grqq{} klicken können, um an ihren Workflows weiter zu arbeiten. 
            \item Nutzer kann normal weiterarbeiten
        \end{itemize}
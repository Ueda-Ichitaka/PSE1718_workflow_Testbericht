\chapter{Browser Support}

Um den von uns gesetzten Qualitätsstandard zu genügen wurde die Applikation ausgiebig von uns getestet.
Uns war dabei wichtig, dass der Benutzer unabhängig von seinem Browser reibungslos an seinen Workflows arbeiten kann.
\newline
\newline
Die Applikation wurde mit den folgenden Browsern getestet
\begin{itemize}
  \item Google Chrome Version 65
  \item Mozilla Firefox Version 58
  \item Safari Version 10
  \item Microsoft Edge Version 41
\end{itemize}


\chapter{Einheitlichkeit des Codes}
Da es sowohl im Laufe der Implementierungsphase, als auch im Laufe der Testphase streng den Standards gefolgt wurde, ist der Quellcode normalisiert und einheitlich. Serverseitig haben alle Teammitglieder dem \gls{PEP}8 Standard gefolgt und zusätzlich noch die gleiche Entwicklungsumgebung benutzt, was ausschließt, dass es zum Beispiel verschiedene Stile der Variablen-/Methoden-/Klassenbenenung gewählt wurden. Clientseitig wurde das TSLint Werkzeug für dieselbe Zwecke verwendet.